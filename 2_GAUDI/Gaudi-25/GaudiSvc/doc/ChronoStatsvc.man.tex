\documentclass{lhcbnote}

\usepackage{a4}
\usepackage{times}
\usepackage{listings}
\usepackage{lscape}
\usepackage{longtable}

\newcommand{\bfsc}         {\scshape\bfseries}
\newcommand{\bftt}         {\ttfamily\bfseries}
\newcommand{\bfit}         {\itshape\bfseries}
\newcommand{\scbf}         {\scshape\bfseries}
\newcommand{\slbf}         {\slshape\bfseries}
\newcommand{\itbf}         {\itshape\bfseries}
\newcommand{\ttbf}         {\ttfamily\bfseries}

\renewcommand{\sc  }       {\scshape}
\renewcommand{\sl}         {\slshape}
\renewcommand{\it}         {\itshape}
\renewcommand{\tt}         {\ttfamily}


\begin{document}

\lstset{language=[ANSI]C++}
%\lstset{language=[Visual]C++}
\lstset{indent=15mm}
\lstset{labelstep=1}
\lstset{labelstyle=\tt\tiny}

\doctyp{Internal Note}
\dociss{1}
\docrev{2}
\docref{LHCb-COMP-Offline-2004-064}
\doccre{December 1, 1999}
\docmod{\today}


                
\title{Chrono \& Stat Service}
\author{Vanya Belyaev\footnote{E-mail:{\tt Ivan.Belyaev@itep.ru}} \\ 
  {\it LAPP/Annecy \& ITEP/Moscow}}
\maketitle 

\begin{abstract}
The detailed description of standard Chrono \& Stat Service
is presenetd. The Service is designed for simple code profiling 
and statistical monitoring.
\end{abstract}

\tableofcontents 

\chapter{Chrono \& Stat Service and {\bftt{IChronoStatSvc}}  Interface}  

Chrono \& Stat Service is one of standard components directly 
visible by Algorithm object. 
This service is designed to fulfil two different task:

\begin{itemize}
 \item chrono profiling  of  the  codes  ("Chrono" part)  
 \item simple statistical monitoring of useful quantities ("Stat" part)  
\end{itemize} 

\section{Access to the Chrono \& Stat Service}  
Communication with Chrono \& Stat Service is performed via an 
abstract  interface {\bftt{IChronoStatSvc}}.  
Each Algorithm object\footnote{The method is also available 
  through the base class {\tt{GaudiTool}} for Tool object}  has the 
method for accessing to the pointer to an abstract interface of type 
{\it IChronoStatSvc}:  

\begin{lstlisting}{}
  // The standard Chrono & Stat Service, 
  // Return a pointer to the service if present 
  IChronoStatSvc*   chronoSvc() const ; 
\end{lstlisting} 

For accessing the Chrono \& Stat Service outside the Algorithm 
object, one should use standard method of service location, using 
either the name of the service {\it "ChronoStatSvc"} or the unique
service ID: {\bftt{ extern const InterfaceID\& IID\_IChronoStatSvc}}. 

\section{{\bftt{IChronoStatSvc}} abstract interface}

The most important methods from abstract
{\bftt{IChronoStatSvc}} interface are listed here:
 
\begin{lstlisting}{}
 class IChronoStatSvc : virtual public IService  
{  
  ....
public:
  ...
  // the actual type for Chrono Tag 
  typedef    std::string     ChronoTag  ;
  // the actual type for Stat   Tag 
  typedef    std::string     StatTag    ;
  // type of the Flag   variable for statistics
  typedef    double          StatFlag   ; 
  // type of the Weight variable for statistics 
  typedef    double          StatWeight ;  
  // Type of the delta-time
  typedef    longlong        ChronoTime ;  
  ... 
  enum ChronoStatus
    { UNKNOWN = 0 , RUNNING , STOPPED } ;
  enum ChronoType 
    { USER    = 0 , KERNEL  , ELAPSED } ;
  ... 

  virtual StatusCode   chronoStart  
          ( const ChronoTag&  t ) = 0;
  virtual StatusCode   chronoStop   
          ( const ChronoTag&  t ) = 0;
  virtual ChronoTime   chronoDelta  
          ( const ChronoTag&  t , 
            ChronoType        f ) = 0;
  virtual StatusCode   chronoPrint  
          ( const ChronoTag&  t ) = 0; 
  virtual ChronoStatus chronoStatus 
          ( const ChronoTag&  t ) = 0;
  
  virtual StatusCode   stat         
          ( const StatTag&    t , 
            const StatFlag&   f , 
            const StatWeight& w ) = 0;
  virtual StatusCode   statPrint    
          ( const StatTag&    t ) = 0;


  virtual const ChronoEntity* chrono 
          ( const ChronoTag& t ) const = 0 ;
  virtual const StatEntity*   stat   
          ( const StatTag&   t ) const = 0 ;

  ...
};
\end{lstlisting} 

\section{Code profiling} 

Profiling is performed by using {\bftt{chronoStart}} and 
{\bftt{chronoStop}} methods inside the codes to be profiled, e.g:

\begin{lstlisting}{}
  /// ...
  IChronoStatSvc* svc = ...  
  ///  start 
  svc->startChrono( "Some Tag" ); 
  /// here some user code are placed:    
  ... 
  /// stop 
  svc->stopChrono( "SomeTag" ); 
\end{lstlisting} 

The profiling informations accumulates 
under the {\it{tag}} name, given as an argument for these method. 
The Service performs the measurement of the 
time between subsequent calls of {\bftt{startChrono}}
 method and {\bftt{endChrono}} method with the same {\it{tag}}. 
The later is important, since from the sequence of calls

\begin{lstlisting}{} 
svc->endChrono("Tag"); 
svc->endChrono("Tag"); 
svc->startChrono("Tag"); 
svc->startChrono("Tag");
svc->endChrono("Tag");
svc->endChrono("Tag"); 
svc->startChrono("Tag"); 
svc->startChrono("Tag"); 
svc->endChrono("Tag"); 
\end{lstlisting} 

 only the elapsed time between 3 and 5 lines and the 
elapsed time between 7 and 9 lines would be accumulated.    

The information could be printed either directly 
using {\bftt{printChrono}} method of standard 
Chrono \& Stat Service, or could be printed in the 
final table of profiling information. 

The detailed information is available through 
the helper class {\bftt{ChronoEntity}}. 

\begin{lstlisting}{} 
// get the full information 
const ChronoEntity* entity = svc->chrono("Tag");
\end{lstlisting} 

The class {\bftt{ChronoEntity}}, defined in the file 
{\bftt{GaudiKernel/ChronoEntity.h}} is equipped with set 
of methods which allow detailed run-time inspection 
of profiling information. The major public methods
are listed here:

\begin{lstlisting}{}
class ChronoEntity 
{
  ...
  public:
  /// return the status of chrono
  IChronoStatSvc::ChronoStatus     status() const ;
   /// number of chrono measurements
  unsigned long        nOfMeasurements   () const ;

  /// minimal measurement for ``user'' time
  long double          uMinimalTime      () const ;
  /// minimal measurement for ``kernel'' time
  long double          kMinimalTime      () const ;
  /// minimal measurement for ``ellapsed'' time
  long double          eMinimalTime      () const ;

  /// maximal measurement for ``user'' time
  long double          uMaximalTime      () const ;
  /// maximal measurement for ``kernel'' time
  long double          kMaximalTime      () const ; 
  /// maximal measurement for ``ellapsed'' time
  long double          eMaximalTime      () const ;

  /// total ``user'' time
  long double          uTotalTime        () const ;
  /// total ``Kernel'' time
  long double          kTotalTime        () const ;
  /// total ``Elapsed'' time
  long double          eTotalTime        () const ;

  /// total time (``user'' + ``kernel'') 
  long double          totalTime         () const ;

  /// sum of squared ``user''   time intervals
  long double          uStatistics       () const ;
  /// sum of squared ``Kernel'' time intervals
  long double          kStatistics       () const ;
  /// sum of squared ``Elapsed'' time intervals
  long double          eStatistics       () const ;
  
  /// average ``User''   Time
  long double          uMeanTime         () const ;
  /// average ``Kernel'' Time
  long double          kMeanTime         () const ;
  /// average ``Elapsed''   Time
  long double          eMeanTime         () const ;

  /// r.m.s ``User'' Time
  long double          uRMSTime          () const ;
  /// r.m.s ``Kernel'' Time
  long double          kRMSTime          () const ;
  /// r.m.s ``Elapsed'' Time
  long double          eRMSTime          () const ;

  /// error in mean ``User''   time
  long double          uMeanErrorTime    () const ;
  /// error in mean ``Kernel'' time
  long double          kMeanErrorTime    () const ;
  /// error in mean ``Elapsed''   time
  long double          eMeanErrorTime    () const ;
  
  /// print the chrono;
  std::string          outputUserTime    () const ; 
  /// print the chrono;
  std::string          outputSystemTime  () const ; 
  /// print the chrono;
  std::string          outputElapsedTime () const ;

  ...
};
\end{lstlisting} 

This class allows to inspect the total, minimal, maximal, 
mean, RMS and uncertainty in mean evaluation 
for ``User'', ``Kernel'' and ``Elapsed'' time measurements. 
Also the helper statistics of accumulated corresponding 
squared time intervals  is available, which allow to evaluate
other statictical quantities e.g. differents statistical 
moments. 
 
\section{Statistical monitoring} 

Statistical monitoring is performed by using the {\bftt{stat}} method 
inside user code:
 
\begin{lstlisting}{}
/// ... Flag and Weight to be accumulated:
svc->stat( " Number of Tracks " , Flag , Weight ); 
\end{lstlisting} 

The statistical information contains the 
"accumulated" {\it flag }, which is the sum of all 
{\tt{Flag}}s for the given tag, and the 
"accumulated" {\it weight}, which is the product of all 
{\tt{Weight}}s for the given tag.   
The information is printed in the final table of 
statistics. 

The detailed information is available through 
the helper class {\bftt{StatEntity}}. 

\begin{lstlisting}{} 
// get the full information 
const StatEntity* entity = svc->stat("Tag");
\end{lstlisting} 

The major pubulic methods of class {\bftt{StatEntity}}, defined
in the file {\bftt{GaudiKernel/StatEntity.h}} are listed here:

\begin{lstlisting}{}
class StatEntity
{
  ...
  public:
  /// number of measurements  
  const size_t        nEntries           () const ;

  /// minimal flag
  const StatFlag      flagMin            () const ;
  /// maximal flag
  const StatFlag      flagMax            () const ;
  /// accumulated "flag"
  const StatFlag      flag               () const ;
  /// accumulated "flag squared"
  const StatFlag      flag2              () const ;
  /// mean value of flag 
  const StatFlag      flagMean           () const ; 
  /// r.m.s of flag 
  const StatFlag      flagRMS            () const ;
  /// error in mean value of flag 
  const StatFlag      flagMeanErr        () const ;

  /// minimal "weight"
  const StatWeight    weightMin          () const ;
  /// maximal "weight"
  const StatWeight    weightMax          () const ;
  /// accumulated "weight"
  const StatWeight    weight             () const ;
  /// "mean harmonic" weight 
  const StatWeight    weightHarmonicMean () const ;
  
  /// output of StatEntity 
  const std::string   stringOutPut       () const ;
  ...
};
\end{lstlisting}

This class allows to inspect the total, minimal, maximal, 
mean, RMS and uncertainty in mean evaluation 
for ``Flag'' variable. Also the helper statistics of 
accumulated squared ``Flag'' is available.
For ``Weight'' variable the accumulated ``Weight'', 
minimal and maximal values and the harmonic mean is available. 


In some sense the profiling could be considered as 
statistical monitoring with variable {\tt{Flag}} equals 
to elapsed time of the process.   
 
\chapter{{\bftt{Chrono}} and {\bftt{Stat}} helper classes}

 To simplify the usage  of Chrono \& Stat Service, 
two helper classes were developed: {\bftt{Chrono}} 
and {\bftt{Stat}}. 
Using these utilities, one hides the communications with 
Chrono \& Stat Service and provides more friendly environment. 

\section{\bftt{Chrono}} 
{\bftt{Chrono}} is a small helper class, which 
invokes {\bftt{startChrono}} method of Chrono \& Stat Service 
in the  constructor  and invokes {\bftt{endChrono}} 
method in the  destructor. It must be used as 
{\it automatic local object}.  It performs the profiling of the 
codes since its own creation till the end of the current scope, e.g: 

\begin{lstlisting}{}
/// ... 
{ // begin of the scope 
   Chrono chrono( svc , "ChronoTag" ) ;
   /// some codes:
    ...
   ///
} // end of the scope
/// ...  
\end{lstlisting}   

For usage of {\bftt{Chrono}} utility one must include the file 
{\bftt{GaudiKernel/Chrono.h}}.

If the Chrono \& Stat Service  is not accesible \verb+svc=0+,
the {\bftt{Chrono}} object does nothing.
 
\section{{\bftt{Stat}}}

{\bftt{Stat}} is a small helper class, which 
invokes {\bftt{stat}} method of Chrono \& Stat Service 
in the  constructor:

\begin{lstlisting}{}
/// ... 
Stat stat( svc , "StatTag" , Flag , Weight ) ;
/// ...  
\end{lstlisting} 

For usage of {\bftt{Stat}} utility one must include the file 
{\bftt{GaudiKernel/Stat.h}}

If the Chrono \& Stat Service  is not accesible \verb+svc=0+,
the {\bftt{Stat}} object does nothing.

\chapter{Properties of standard {\bftt{ChronoStatSvc}} service }
   
\section{Properties for profiling  } 
Profiling properties of the standard Chrono \& Stat Service 
which are configurable via {\it job options file} and their 
default values are:

\begin{lstlisting}{} 
// deside if the final printout should be performed;  
ChronoStatSvc.ChronoPrintOutTable = true; 
// if User   Time information to be printed?;
ChronoStatSvc.PrintUserTime    = true       ; 
// if System Time information to be printed?;
ChronoStatSvc.PrintSystemTime  = false      ;
// define the destination of the table to be printed;
ChronoStatSvc.ChronoDestinationCout = false;
// print level for profiling  ( according to MSG::Level)  
ChronoStatSvc.ChronoPrintLevel      = 3 ; 
// if printout is to be performed, 
// should one take care about some ordering?  
ChronoStatSvc.ChronoTableToBeOrdered = true ;
\end{lstlisting}
 
Tables are ordered according the total elapsed time. 

\section{Properties for statistics } 
Statistical properties of the standard Chrono \& Stat Service 
which are configurable via{\it job options file} and their default values 
are:

\begin{lstlisting}{} 
// deside if the final printout should be performed;  
ChronoStatSvc.StatPrintOutTable = true; 
// define the destination of the table to be printed;
ChronoStatSvc.StatDestinationCout = false;
// print level for profiling  ( according to MSG::Level)  
ChronoStatSvc.StatPrintLevel      = 3 ; 
// if printout is to be performed, 
// should one take care about some ordering?  
ChronoStatSvc.StatTableToBeOrdered = true ;
\end{lstlisting}


\chapter{Performance consideration  } 
Implementation of Chrono \& Stat Service used two {\bftt{ std::map}} 
containers. Therefore for very frequent calls it could cause the 
performance penalty.  Usually this penalty  is negligible with respect to the 
elapsed time of algorithms, but it is worth to avoid both the direct usage 
of Chrono \& Stat Service as well as the  usage of it 
through {\bftt{Chrono}} or {\bftt{Stat}} utilities inside the 
internal loops:

\begin{lstlisting}{}
/// ...
{  /// begin of the scope 
Chrono chrono( svc , "Good Chrono");  /// OK
long double a = 0 ; 
for(  long i = 0 ; i < 1000000 ; ++i ) 
 {
 Chrono chrono( svc , "Bad Chrono");   /// not OK
 /// some code :
   a += sin( cos( sin( cos( (long double) i ) ) ) ); 
 ///  end of codes
Stat   stat  ( svc , "Bad Stat", a ); /// not OK 
 }  
Stat   stat  ( svc , "Good Stat", a); /// OK 
} ///  end of the scope!
/// ...  
\end{lstlisting}  

\end{document} 